\documentclass[
%DIF LATEXDIFF DIFFERENCE FILE
%DIF DEL /tmp/NA76orWatI/latexdiff-vc-HEAD/thesis.tex   Wed Dec  2 15:48:22 2020
%DIF ADD thesis.tex                                     Thu Dec  3 08:29:26 2020
  sotsuron]{kuee}


\usepackage[dvipdfmx]{hyperref}
\usepackage{pxjahyper}
\usepackage{longtable,booktabs}
\usepackage{listings}
\usepackage{xcolor}


\usepackage[dvipdfmx]{graphicx,grffile}
\makeatletter
\def\maxwidth{\ifdim\Gin@nat@width>\linewidth\linewidth\else\Gin@nat@width\fi}
\def\maxheight{\ifdim\Gin@nat@height>\textheight\textheight\else\Gin@nat@height\fi}
\makeatother
% Scale images if necessary, so that they will not overflow the page
% margins by default, and it is still possible to overwrite the defaults
% using explicit options in \includegraphics[width, height, ...]{}
\setkeys{Gin}{width=\maxwidth,height=\maxheight,keepaspectratio}


\lstset{
    basicstyle=\ttfamily,
    keywordstyle=\color[RGB]{33,74,135}\bfseries,
    stringstyle=\color[RGB]{79,153,5},
    commentstyle=\color[RGB]{143,89,2}\itshape,
    numberstyle=\footnotesize,
    numbers=left,
    stepnumber=1,
    numbersep=15pt,
    backgroundcolor=\color[RGB]{251,251,251},
    frame=single,
    frameround=ffff,
    framesep=5pt,
    rulecolor=\color[RGB]{148,150,152}, 
    breaklines=true,
    breakautoindent=true,
    breakatwhitespace=true,
    breakindent=25pt,
    showspaces=false,
    showstringspaces=false,
    showtabs=false,
    tabsize=2,
    captionpos=b,
    linewidth=\textwidth,
}
% ここまで


%DIF 51c51
%DIF < \title{LaTeX を用いた修論$\cdot$卒論の執筆}
%DIF -------
\title{LaTeX2 を用いた修論$\cdot$卒論の執筆} %DIF > 
%DIF -------
\etitle{Usage of The LaTeX Style File for KUEE}
\author{岸 直輝}
\eauthor{Jiro Denki}
\professor{電気 太郎 教授}
\date{令和2年4月14日}
\makeatletter
\@ifpackageloaded{subfig}{}{\usepackage{subfig}}
\@ifpackageloaded{caption}{}{\usepackage{caption}}
\captionsetup[subfloat]{margin=0.5em}
\AtBeginDocument{%
\renewcommand*\figurename{図}
\renewcommand*\tablename{表}
}
\AtBeginDocument{%
\renewcommand*\listfigurename{List of Figures}
\renewcommand*\listtablename{List of Tables}
}
\@ifpackageloaded{float}{}{\usepackage{float}}
\floatstyle{ruled}
\@ifundefined{c@chapter}{\newfloat{codelisting}{h}{lop}}{\newfloat{codelisting}{h}{lop}[chapter]}
\floatname{codelisting}{コード}
\newcommand*\listoflistings{\listof{codelisting}{List of Listings}}
\makeatother

\date{}
%DIF PREAMBLE EXTENSION ADDED BY LATEXDIFF
%DIF UNDERLINE PREAMBLE %DIF PREAMBLE
\RequirePackage[normalem]{ulem} %DIF PREAMBLE
\RequirePackage{color}\definecolor{RED}{rgb}{1,0,0}\definecolor{BLUE}{rgb}{0,0,1} %DIF PREAMBLE
\providecommand{\DIFaddtex}[1]{{\protect\color{blue}\uwave{#1}}} %DIF PREAMBLE
\providecommand{\DIFdeltex}[1]{{\protect\color{red}\sout{#1}}}                      %DIF PREAMBLE
%DIF SAFE PREAMBLE %DIF PREAMBLE
\providecommand{\DIFaddbegin}{} %DIF PREAMBLE
\providecommand{\DIFaddend}{} %DIF PREAMBLE
\providecommand{\DIFdelbegin}{} %DIF PREAMBLE
\providecommand{\DIFdelend}{} %DIF PREAMBLE
%DIF FLOATSAFE PREAMBLE %DIF PREAMBLE
\providecommand{\DIFaddFL}[1]{\DIFadd{#1}} %DIF PREAMBLE
\providecommand{\DIFdelFL}[1]{\DIFdel{#1}} %DIF PREAMBLE
\providecommand{\DIFaddbeginFL}{} %DIF PREAMBLE
\providecommand{\DIFaddendFL}{} %DIF PREAMBLE
\providecommand{\DIFdelbeginFL}{} %DIF PREAMBLE
\providecommand{\DIFdelendFL}{} %DIF PREAMBLE
%DIF HYPERREF PREAMBLE %DIF PREAMBLE
\providecommand{\DIFadd}[1]{\texorpdfstring{\DIFaddtex{#1}}{#1}} %DIF PREAMBLE
\providecommand{\DIFdel}[1]{\texorpdfstring{\DIFdeltex{#1}}{}} %DIF PREAMBLE
\newcommand{\DIFscaledelfig}{0.5}
%DIF HIGHLIGHTGRAPHICS PREAMBLE %DIF PREAMBLE
\RequirePackage{settobox} %DIF PREAMBLE
\RequirePackage{letltxmacro} %DIF PREAMBLE
\newsavebox{\DIFdelgraphicsbox} %DIF PREAMBLE
\newlength{\DIFdelgraphicswidth} %DIF PREAMBLE
\newlength{\DIFdelgraphicsheight} %DIF PREAMBLE
% store original definition of \includegraphics %DIF PREAMBLE
\LetLtxMacro{\DIFOincludegraphics}{\includegraphics} %DIF PREAMBLE
\newcommand{\DIFaddincludegraphics}[2][]{{\color{blue}\fbox{\DIFOincludegraphics[#1]{#2}}}} %DIF PREAMBLE
\newcommand{\DIFdelincludegraphics}[2][]{% %DIF PREAMBLE
\sbox{\DIFdelgraphicsbox}{\DIFOincludegraphics[#1]{#2}}% %DIF PREAMBLE
\settoboxwidth{\DIFdelgraphicswidth}{\DIFdelgraphicsbox} %DIF PREAMBLE
\settoboxtotalheight{\DIFdelgraphicsheight}{\DIFdelgraphicsbox} %DIF PREAMBLE
\scalebox{\DIFscaledelfig}{% %DIF PREAMBLE
\parbox[b]{\DIFdelgraphicswidth}{\usebox{\DIFdelgraphicsbox}\\[-\baselineskip] \rule{\DIFdelgraphicswidth}{0em}}\llap{\resizebox{\DIFdelgraphicswidth}{\DIFdelgraphicsheight}{% %DIF PREAMBLE
\setlength{\unitlength}{\DIFdelgraphicswidth}% %DIF PREAMBLE
\begin{picture}(1,1)% %DIF PREAMBLE
\thicklines\linethickness{2pt} %DIF PREAMBLE
{\color[rgb]{1,0,0}\put(0,0){\framebox(1,1){}}}% %DIF PREAMBLE
{\color[rgb]{1,0,0}\put(0,0){\line( 1,1){1}}}% %DIF PREAMBLE
{\color[rgb]{1,0,0}\put(0,1){\line(1,-1){1}}}% %DIF PREAMBLE
\end{picture}% %DIF PREAMBLE
}\hspace*{3pt}}} %DIF PREAMBLE
} %DIF PREAMBLE
\LetLtxMacro{\DIFOaddbegin}{\DIFaddbegin} %DIF PREAMBLE
\LetLtxMacro{\DIFOaddend}{\DIFaddend} %DIF PREAMBLE
\LetLtxMacro{\DIFOdelbegin}{\DIFdelbegin} %DIF PREAMBLE
\LetLtxMacro{\DIFOdelend}{\DIFdelend} %DIF PREAMBLE
\DeclareRobustCommand{\DIFaddbegin}{\DIFOaddbegin \let\includegraphics\DIFaddincludegraphics} %DIF PREAMBLE
\DeclareRobustCommand{\DIFaddend}{\DIFOaddend \let\includegraphics\DIFOincludegraphics} %DIF PREAMBLE
\DeclareRobustCommand{\DIFdelbegin}{\DIFOdelbegin \let\includegraphics\DIFdelincludegraphics} %DIF PREAMBLE
\DeclareRobustCommand{\DIFdelend}{\DIFOaddend \let\includegraphics\DIFOincludegraphics} %DIF PREAMBLE
\LetLtxMacro{\DIFOaddbeginFL}{\DIFaddbeginFL} %DIF PREAMBLE
\LetLtxMacro{\DIFOaddendFL}{\DIFaddendFL} %DIF PREAMBLE
\LetLtxMacro{\DIFOdelbeginFL}{\DIFdelbeginFL} %DIF PREAMBLE
\LetLtxMacro{\DIFOdelendFL}{\DIFdelendFL} %DIF PREAMBLE
\DeclareRobustCommand{\DIFaddbeginFL}{\DIFOaddbeginFL \let\includegraphics\DIFaddincludegraphics} %DIF PREAMBLE
\DeclareRobustCommand{\DIFaddendFL}{\DIFOaddendFL \let\includegraphics\DIFOincludegraphics} %DIF PREAMBLE
\DeclareRobustCommand{\DIFdelbeginFL}{\DIFOdelbeginFL \let\includegraphics\DIFdelincludegraphics} %DIF PREAMBLE
\DeclareRobustCommand{\DIFdelendFL}{\DIFOaddendFL \let\includegraphics\DIFOincludegraphics} %DIF PREAMBLE
%DIF LISTINGS PREAMBLE %DIF PREAMBLE
\lstdefinelanguage{codediff}{ %DIF PREAMBLE
  moredelim=**[is][\color{red}]{*!----}{----!*}, %DIF PREAMBLE
  moredelim=**[is][\color{blue}]{*!++++}{++++!*} %DIF PREAMBLE
} %DIF PREAMBLE
\lstdefinestyle{codediff}{ %DIF PREAMBLE
	belowcaptionskip=.25\baselineskip, %DIF PREAMBLE
	language=codediff, %DIF PREAMBLE
	basicstyle=\ttfamily, %DIF PREAMBLE
	columns=fullflexible, %DIF PREAMBLE
	keepspaces=true, %DIF PREAMBLE
} %DIF PREAMBLE
%DIF END PREAMBLE EXTENSION ADDED BY LATEXDIFF

\begin{document}
\maketitle

\begin{eabstract}This document briefly explains the usage of the\end{eabstract}

\def\lstlistingname{ソースコード}

{
        \setcounter{tocdepth}{2}
    \tableofcontents
  }
\hypertarget{ux306fux3058ux3081ux306b}{%
\chapter{はじめに}\label{ux306fux3058ux3081ux306b}}

\label{chap:intro}

この文書は,京都大学電気電子工学科の修論 \(\cdot\) 卒論作成用 \LaTeX
クラスファイルの利用方法についての説明書です.

修論\(\cdot\)卒論の正確なフォーマットの指定は,事務室から配布される手引
を参照してください.

\cite{GuideBook}

adfa (晴彦 and 裕介 2017, pp)

あいうえお\footnote{あいうえおあ}

\hypertarget{ux4feeux8ad6cdotux5352ux8ad6ux30afux30e9ux30b9ux30d5ux30a1ux30a4ux30ebux306eux5229ux7528ux65b9ux6cd5}{%
\chapter{\texorpdfstring{修論\(\cdot\)卒論クラスファイルの利用方法}{修論\textbackslash cdot卒論クラスファイルの利用方法}}\label{ux4feeux8ad6cdotux5352ux8ad6ux30afux30e9ux30b9ux30d5ux30a1ux30a4ux30ebux306eux5229ux7528ux65b9ux6cd5}}

\hypertarget{ux30a4ux30f3ux30b9ux30c8ux30fcux30eb}{%
\section{インストール}\label{ux30a4ux30f3ux30b9ux30c8ux30fcux30eb}}

配布キットには,表 \ref{tab:kit} のファイルが含まれています.

\begin{table}
  \caption{配布キットのファイル一覧}\label{tab:kit}
  \begin{center}
    \begin{tabular}{ll}
      \verb+kuee.cls+ & 修論$\cdot$卒論用 \LaTeX2e{} クラスファイル       \\
      \verb+kueethesis.bst+ & 修論$\cdot$卒論用 文献スタイルファイル \\
      \verb+sample.tex+ & 使用の手引(このドキュメント)を作るファイル        \\
      \verb+sample.bib+ & 使用の手引の参考文献を収めたファイル              \\
    \end{tabular}
  \end{center}
\end{table}

このクラスファイルを使用するため, kuee.cls と kueethesis.bst
を,環境変数 TEXINPUTS で指定されたディレ
クトリ,または修論\(\cdot\)卒論の原稿と同じディレクトリにコピーして下さい.

配布キットの文字コードは Unicode になっています.利用環境に応じて,適
切に文字コードを変換してください.

\hypertarget{ux30c9ux30adux30e5ux30e1ux30f3ux30c8ux30b9ux30bfux30a4ux30eb}{%
\section{ドキュメントスタイル}\label{ux30c9ux30adux30e5ux30e1ux30f3ux30c8ux30b9ux30bfux30a4ux30eb}}

ドキュメントスタイルは,オプションとして指定します.修論の場合は
shuuron,卒論の場合は sotsuron
を用います.例えば,卒論の場合は次のように指定して下さい.

\begin{quote}
  \begin{verbatim}
\documentclass[sotsuron]{kuee}
\end{verbatim}
\end{quote}

指定を省略すると,修論用のスタイルが選択されます.

\hypertarget{ux6587ux5b57ux6570ux884cux6570ux306eux8a2dux5b9a}{%
\section{文字数,行数の設定}\label{ux6587ux5b57ux6570ux884cux6570ux306eux8a2dux5b9a}}

1行の文字数と1ページの行数を指定する場合は,\verb+\begin{document}+ よ
り先に次のように指定します.これは,デフォルトと同じ1行36文字,1ページ
32行の設定例です.

\begin{quote}
  \begin{verbatim}
\charsinline{36}
\linesinpage{32}
\end{verbatim}
\end{quote}

ただし,\TeX のページ分割のためにすべてのページが必ずしも設定通りの行
数にはなりません.ASCII \TeX では句読点のカーニングの伸縮のため,すべ
ての行が必ずしも設定通りの文字数にはなりません.なお,1行の文字数はNTT
\TeX では最大38文字,ASCII \TeX では最大37文字で,これを越えると隣合
う文字同士が重なってしまいます.

\hypertarget{ux8868ux7d19}{%
\section{表紙}\label{ux8868ux7d19}}

表紙は \verb+\maketitle+ コマンドによって出力されます\footnote{表紙ペー
  ジのページ番号は0ですが,出力されません.}.

表紙を出力するコマンド \verb+\maketitle+ よりも先に,タイトル(日本語お
よび英語),著者氏名(日本語および英語)などを次のように指定する必要があ
ります.

\begin{quote}
  \begin{verbatim}
\title{\LaTeX を用いた修論$\cdot$卒論の執筆}
\etitle{Usage of The \LaTeX{} Style File for KUEE}
\author{電気 次郎}
\eauthor{Jiro Denki}
\professor{電気 太郎 教授}
\date{平成13年12月18日}
\end{verbatim}
\end{quote}

なお,英語タイトルおよび英語著者氏名は,表紙ではなく,英文要旨を出力す
る時に用いられます.

上記の例のように,研究科(学部)と専攻(学科)の指定を省略すると,修論作成
時には,工学研究科 電気工学専攻が指定されたと見なされます.卒論作成時に
は,工学部 電気電子工学科が指定されたと見なされます.

研究科(学部)を指定する場合は \verb+\course+ コマンドを,専攻(学科)を指
定する場合は \verb+\department+ コマンドを,以下のように使用してくださ
い.

\begin{quote}
  \begin{verbatim}
\course{京都大学大学院情報学研究科}
\department{知能情報学専攻}
\end{verbatim}
\end{quote}

\hypertarget{ux82f1ux6587ux8981ux65e8}{%
\section{英文要旨}\label{ux82f1ux6587ux8981ux65e8}}

英文要旨は,\verb+eabstract+ 環境を用いて記述します.

\hypertarget{ux76eeux6b21}{%
\section{目次}\label{ux76eeux6b21}}

目次は \verb+\tableofcontents+ コマンドによって出力されます\footnote
{目次ページのページ番号は,ローマ数字で出力されます.}.謝辞,参考文献,
付録なども目次に掲載されます.

\hypertarget{ux672cux6587}{%
\section{本文}\label{ux672cux6587}}

本文は通常の\LaTeX のテキストとして記述します\footnote{本文ページのペー
  ジ番号は1から始まり,アラビア数字で出力されます.}.
いくつかの点でj-report/jreport/reportスタイルとの違いがあります.

\hypertarget{ux7ae0ux984c}{%
\section{章題}\label{ux7ae0ux984c}}

章題は次のように出力されます.

章題が1行に収まり切らない場合には次のように改行されて出力されます.

\hypertarget{ux811aux6ce8}{%
\subsection{脚注}\label{ux811aux6ce8}}

脚注は章ごとにカウントされ,マークは\(^{*}\), \(^{**}\), \(^{***}\),
\(^{\dagger}\), \(^{\dagger\dagger}\), \(^{\dagger\dagger\dagger}\),
\(^{\ddagger}\), \ldots のようになります.

\hypertarget{ux56f3ux8868}{%
\subsection{図表}\label{ux56f3ux8868}}

通常の \LaTeX を利用する場合と同様,本文中の適当な場所に記述して下さい.
全ての図表は,\TeX によって自動的に論文の末尾に移動されます.例えば, 図
\ref{fig:example} は,この段落の直後で定義されていますが,実際の整
形結果では論文末尾に移動しているはずです.

\begin{figure}
  \begin{center}
    \unitlength=1mm
    \begin{picture}(100,100)(-50,-50)
      \put(0,-50){\vector(0,1){100}}
      \put(-50,0){\vector(1,0){100}}
      \put(0,0){\makebox(0,0)[rt]{$o$}}
      \put(50,0){\makebox(0,0)[lt]{$x$}}
      \put(0,50){\makebox(0,0)[rb]{$y$}}
      \put(0,0){\vector(2,1){30}}
      \put(30,15){\makebox(0,0)[lt]{$\vec{a}$}}
      \put(0,0){\vector(1,2){15}}
      \put(15,30){\makebox(0,0)[lt]{$\vec{b}$}}
      \thicklines
      \put(0,0){\vector(1,1){45}}
      \put(45,45){\makebox(0,0)[lt]{$\vec{a}+\vec{b}$}}
    \end{picture}
  \end{center}
  \caption{figure 環境の例}
  \label{fig:example}
 \end{figure}

大量の図表を張り付けると,以下のようなエラーが発生することがあります.

\begin{quote}
  \begin{verbatim}
! LaTeX Error: Too many unprocessed floats.
\end{verbatim}
\end{quote}

\LaTeX が図表を組み版する時は,前後の文章の量を見ながらオプションで指
定された条件に合う場所が出てくるまでメモリーに図表を保存しています。上
記エラーは,図表が数ページにわたって連続して現われ,メモリーが不足する
と発生します.このエラーが発生した時は,適当な位置に
\verb+\clearfigurepage+ コマンドを挿入してください.このコマンドは,図
表ページを指定された個所で強制的に分割し,組版処理を行うように指示しま
す.

\hypertarget{ux8b1dux8f9e}{%
\section{謝辞}\label{ux8b1dux8f9e}}

謝辞は,\verb+acknowledgements+ 環境を用いて記述します.

\hypertarget{ux76f8ux4e92ux53c2ux7167}{%
\section{相互参照}\label{ux76f8ux4e92ux53c2ux7167}}

\label{cross_reference}
相互参照は,通常の\LaTeX\{\}文書と同様に,\verb+\label+ コマンドと
\verb+\ref+ コマンドを用いて行います.例えば,章番号を参照する場合には,
以下のように \verb+\chapter+ コマンドの直後に \verb+\label+ コマンドを
配置してラベルを宣言します.

\begin{quote}
  \begin{verbatim}
\chapter{はじめに}
\label{chap:intro}
\end{verbatim}
\end{quote}

その上で,参照したい箇所に,\verb+\ref+ コマンドを以下のように配置しま
す.

\begin{quote}
  \begin{verbatim}
\ref{chap:intro}章では,本利用説明書の位置づけについて述べています.
\end{verbatim}
\end{quote}

\verb+\ref{chap:intro}+
は実際の章番号に置換されて,以下のように組版されます.

\begin{quote}
  \ref{chap:intro}章では,本利用説明書の位置づけについて述べています.
\end{quote}

詳しくは,本利用説明書内の利用例および,
\LaTeX2e\{\}美文書作成入門\cite{GuideBook}の第10章などを参照してください.

\hypertarget{ux53c2ux8003ux6587ux732e}{%
\section{参考文献}\label{ux53c2ux8003ux6587ux732e}}

参考文献は \verb+thebibliography+ 環境を用いて直接記述するか, BibTeX
システムを用いて作成することができます\footnote{(J)BibTeX
  の使い方については \LaTeX ブック\cite{LaTeX}の付録Bなどを参照.}.

\hypertarget{ux4ed8ux9332}{%
\section{付録}\label{ux4ed8ux9332}}

付録は \verb+\appendix+ コマンドの後に記述します\footnote{付録ページの
  ページ番号は本文から継続し,アラビア数字で出力されます.}.付録の各項
目は \verb+\chapter+ コマンドによって分割して記述します.付録がただ1項
目からなる場合にも項目の始めに \verb+\chapter+ コマンドを用いて項目名
を指定して下さい.

\hypertarget{ux304aux308fux308aux306b}{%
\chapter{おわりに}\label{ux304aux308fux308aux306b}}

\label{chap:conclusion}

\LaTeX\{\}209用スタイルファイルの利用説明書\cite{OldTebiki}には,次のよ
うに書かれていました.

\begin{quote}
  京大電気系学科の修論$\cdot$卒論に\LaTeX が使われ出して4年目になりま
  す.最初のころはPC98上のアスキー日本語Micro-\TeX を使ってちんたらやっ
  ていたものですが,最近ではUnixマシンおよびUnix上のNTT \TeX, ASCII
  \TeX が広く普及し,修論・卒論を\LaTeX で書こうという人はかなり多く
  なっているものと思います.
\end{quote}

すなわち,30年の長きにわたって,\LaTeX\{\}が修論\(\cdot\)卒論の作成に用い
られていることになります.これは,\LaTeX\{\}の論理マークアップという考え
方が,論文作成と親和性が高いことの証明であると,著者は考えます.例えば,
論文作成時には,論旨を明確化するために,章や節を単位として順序を頻繁に
変更する必要が生じます.相互参照(\ref{cross_reference}節)を正しく活用
していれば,どのように順序を変更しても常に正しく番号が付番されるので,
納得がいくまで順序を考えることができます.

このクラスファイルを活用し,皆さんがより良い修論\(\cdot\)卒論を執筆され
ることを願っています.

\begin{acknowledgements}
  オリジナルの \LaTeX{}209 用スタイルファイル kueethesis.sty の配布キットを作成された,傳康晴さんに感謝します.


  図表を論文の末尾に移動する方法について, endfloat.sty を
  参考にさせて頂きました.筆者の James Darrell McCauley さんに感謝しま
  す.また,改造方法について \TeX{} FAQ 掲示板でアドバイスをくださった
  misc さんに感謝します.

  このクラスファイルの初期の版から現在の版に至るまで,各年度の長尾研究
  室(現$\cdot$言語メディア研究室)をはじめ多くの研究室の人たちから,数々
  の貴重なコメントを頂きました.関係者各位に感謝します.
\end{acknowledgements}

\bibliographystyle{kueethesis}
\bibliography{thesis}

\appendix
\chapter{改版履歴}\label{chap:history}
\begin{description}
  \item[1991年] 傳康晴\footnote{工学研究科\ 電子工学専攻 長尾研(当時).
          現在,千葉大学.}が,\LaTeX{}209 用修論$\cdot$卒論スタイ
        ルファイル kueethesis.sty と文献用スタイルファ
        イル kueethesis.bst を作成\cite{OldTebiki}.
  \item[2001年] 土屋雅稔\footnote{情報学研究科 知能情報学専攻 言語メディ
        ア研(当時).現在,豊橋技術科学大学.tsuchiya@tut.jp}が,\LaTeX2e{} 用修論$\cdot$卒論クラスファ
        イル kuee.cls を作成\cite{Tebiki2004}.
        \newcounter{tsuaffil}
        \setcounter{tsuaffil}{\value{footnote}}
  \item[2017年] 土谷亮\footnote{情報学研究科 通信情報システム専攻 小野
        寺研.現在,滋賀県立大学.tsuchiya.a@e.usp.ac.jp} が,参考文献の形式を改良.
  \item[2018年] 土屋雅稔\footnotemark[\value{tsuaffil}]{}が,英文要旨
        に対応するためのコマンドを追加.
\end{description}

\chapter{レイアウト・パラメータ}\label{chap:layout}

デフォルトで利用される本文ページ,図・表ページのレイアウト・パラメータ
はそれぞれ表 \ref{tab:text},\ref{tab:fig} のようになっています.

\begin{table}
  \caption{本文ページのデフォルト・レイアウト}\label{tab:text}
  \begin{center}
    \begin{tabular}{|l|r|}
      \hline
      \verb+\textwidth+ & 424pt \\ \hline
      \verb+\textheight+ & 604pt \\ \hline
      \verb+\oddsidemargin+ & 0.5cm \\ \hline
      \verb+\evensidemargin+ & 0.5cm \\ \hline
      \verb+\topmargin+ & 0pt   \\ \hline
      \verb+\headheight+ & 12pt  \\ \hline
      \verb+\headsep+ & 25pt  \\ \hline
      \verb+\footskip+ & 30pt  \\ \hline
    \end{tabular}
  \end{center}
\end{table}

\begin{table}
  \caption{図・表ページのデフォルト・レイアウト}\label{tab:fig}
  \begin{center}
    \begin{tabular}{|l|r|}
      \hline
      \verb+\textwidth+ & 424pt + 1cm  \\ \hline
      \verb+\textheight+ & 604pt + 67pt \\ \hline
      \verb+\oddsidemargin+ & 0pt          \\ \hline
      \verb+\evensidemargin+ & 0pt          \\ \hline
      \verb+\topmargin+ & 0pt          \\ \hline
      \verb+\headheight+ & 0pt          \\ \hline
      \verb+\headsep+ & 0pt          \\ \hline
      \verb+\footskip+ & 0pt          \\ \hline
    \end{tabular}
  \end{center}
\end{table}

\hypertarget{refs}{}
\leavevmode\hypertarget{ref-GuideBook}{}%
晴彦, and 黒木 裕介. 2017. \emph{LaTeX2e美文書作成入門}. 技術評論社.

\end{document}
